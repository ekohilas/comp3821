\documentclass{article}
\usepackage[margin=1in]{geometry}
\usepackage[utf8]{inputenc}
\usepackage{mathtools}
\usepackage{enumitem}
\usepackage{fancyhdr}
\usepackage{chngcntr}
\usepackage[table,xcrdaw]{xcolor}
\lhead{Evan Kohilas - z5114986}
\rhead{COMP3821 - Assignment 2}
\pagestyle{fancy}
\title{A17S1N1}
\counterwithin*{equation}{section}
\begin{document}
\begin{center}
    \begin{LARGE}
        COMP3121\\
        Assignment 2\\
        A17S1N2\\
        \hrulefill\\
        Evangelos Kohilas\\
        z5114986\\
        \hrulefill
    \end{LARGE}

    \begin{large}
        By submitting this document you are confirming that all the answers are your work and are not take from any other sources unless clearly mentioned.
    \end{large}

\end{center}

\section*{Question 1}
\begin{enumerate}[label=\alph*)]
    \item
        \begin{gather}
            \text{let } A^n
            = \begin{pmatrix} F(n+1) & F(n) \\ F(n) & F(n-1)  \end{pmatrix}
        \end{gather}
        When $n = 1$
        \begin{gather}
            A^1 = \begin{pmatrix} 1 & 1 \\ 1 & 0 \end{pmatrix}
        \end{gather}
        Assume $n = k$
        \begin{gather}
            A^k = \begin{pmatrix} F(k+1) & F(k) \\ F(k) & F(k-1)  \end{pmatrix}
        \end{gather}
        let $n = k + 1$
        \begin{align}
            A^{k+1} &= AA^k \\
            &= \begin{pmatrix} 1 & 1 \\ 1 & 0 \end{pmatrix}\begin{pmatrix} F(k+1) & F(k) \\ F(k) & F(k-1)\end{pmatrix} \\
            &= \begin{pmatrix} F(k+1) + F(k) & F(k) + F(k-1) \\ F(k+1) & F(k)  \end{pmatrix} \\
            &= \begin{pmatrix} F(k+2) & F(k+1) \\ F(k+1) & F(k)  \end{pmatrix}
        \end{align}
        Therefore the formula is true by induction for all $n > 0$.
    \item
        $F(n)$ can be found in $\log n$ matrix multiplications using a recursive algorithm
        \begin{verbatim}
matrix = ((1, 1), (1,0))
func(n):
    if n == 1:
        return matrix
    if n is even:
        return func(n/2)^2
    if n is odd:
        return func(n-1) * matrix
        \end{verbatim}
\end{enumerate}

\section*{Question 2}
\begin{enumerate}[label=\alph*)]
    \item
        First we calculate the Karastuba trick.
        \setcounter{equation}{0}
        \begin{gather}
            (a + b)(c + d) = ac + ad + bd + bc \\
            ad + bd = (a + b)(c + d) - ac - bc
        \end{gather}
        Then we substitute $(2)$ into $(4)$
        \begin{align}
            (a + ib)(c + id) &= ac + adi + bdi + bc \\
            &= ac + i(ad + bd) + bc \\
            &= ac + i((a + b)(c + d) - ac - bc) + bc
        \end{align}
        Thus only requiring 3 real number multiplications.
    \item
        \setcounter{equation}{0}
        First we calculate
        \begin{gather}
            a^2 - b^2 = (a + b)(a - b)
        \end{gather}
        Then we substitute $(1)$ into $(2)$
        \begin{align}
            (a + ib)^2 &= a^2 - b^2 + 2abi\\
            &= (a + b)(a - b) + 2abi
        \end{align}
        Thus only requiring 2 real number multiplications
    \item
        \setcounter{equation}{0}
        By re-arranging by the laws of exponents:
        \begin{gather*}
            (a+ib)^2(c+id)^2 = ((a+ib)(c+id))^2
        \end{gather*}
        Thus from above, we then calculate the middle multiplication using
        3 real number multiplications, and then we find the square as above
        using 2 more real number multiplications.
\end{enumerate}

\section*{Question 3}
Expand $P(x)$ and $Q(x)$ as follows.
\begin{align*}
    & P(x) = a_0 + x^{17}(a_{17} + a_{19}x^{2} + a_{21}x^4 + a_{23}x^6)\\
    & Q(x) = b_0 + x^{17}(b_{17} + b_{19}x^{2} + b_{21}x^4 + b_{23}x^6)
\end{align*}
let $y = x^2$ so that
\begin{align*}
    & R_a(y) = a_{17} + a_{19}y + a_{21}y^2 + a_{23}y^3\\
    & R_b(y) = b_{17} + b_{19}y + b_{21}y^2 + b_{23}y^3
\end{align*}
then
\begin{align*}
    & P(x)Q(x) = a_0b_0 + x^{17}(a_0R_b(x^2) + b_0R_a(x^2)) + x^{34}R_a(x^2)R_b(x^2)
\end{align*}
through brute force, we then calculate
\begin{align*}
    & a_0b_0 \text{ to require 1 multiplication}\\
    & a_0R_b(x^2) \text{ and } b_0R_a(x^2) \text{ to require 4 multiplications each}
\end{align*}
then to multiply $R_a(x^2)R_b(x^2)$ (of degree 3), we require $2(3)+1 = 7$ multiplications using the generalised Karatsuba method.\\
and so we get $1 + 4 + 4 + 7 = 16$ multiplications of large numbers.

\section*{Question 4}
As $P(x)$ has all 15 roots of unity, and $x^{15} - 1$ and $P(x)$ are both of the same degree and are both monic then it follows that
\begin{gather*}
P(x) = x^{15} -1
\end{gather*}

\section*{Question 5}
For any input $(a_0, a_1, a_2, ..., a_{2^n-1})$, we can describe $a_i$'s new position by converting $i$ to $n$ binary places and finding the reversed sequence.
e.g. $6 \rightarrow 110 \rightarrow 011 \rightarrow 3$

\section*{Question 6}
let
\begin{align}
    f_m &= \sum_{i+j=m} (j+1)q_jq_i \\
    p_j &= (j+1)q_j
\end{align}
then substitute $(2)$ into $(1)$
\begin{align}
    & \sum_{i+j=m} p_jq_i = \vec{p} \ast \vec{q}
\end{align}
as $(3)$ is a linear convolution, $f_m$ can be computed in $O(n\log n )$ time by transforming $\vec{p}$ and $\vec{q}$ to a point value representation using FFT, calculating the linear multiplication, and then transforming back to coefficient form using inverse FFT.

\pagebreak
\section*{Question 7}
\begin{enumerate}[label=\alph*)]
    \item Starting at $(0,0)$ any spiral arrangement such as the one below would require $n^2$ queries to find the middle element as the local minimum.
        \begin{center}
\texttt{%
\begin{tabular}{|c|c|c|c|c|}
\hline
\rowcolor[HTML]{E6E6E6}
16                        & 15                        & 14                        & 13 & 12                         \\ \hline
17                        & 18                        & 19                        & 20 & \cellcolor[HTML]{E6E6E6}11 \\ \hline
\cellcolor[HTML]{E6E6E6}2 & \cellcolor[HTML]{E6E6E6}1 & \cellcolor[HTML]{E6E6E6}0 & 21 & \cellcolor[HTML]{E6E6E6}10 \\ \hline
\cellcolor[HTML]{E6E6E6}3 & 24                        & 23                        & 22 & \cellcolor[HTML]{E6E6E6}9  \\ \hline
\rowcolor[HTML]{E6E6E6}
4                         & 5                         & 6                         & 7  & 8                          \\ \hline
\end{tabular}
}
        \end{center}
    \item Using the fact that a surface attains its minimal height either along its edges or in the interior, we can develop a divide-and-conquer algorithm by first finding the minimal height along a grid's boundary rows \& columns, and center rows \& columns as follows for an odd and even $n$.
\begin{center}
\texttt{%
\begin{tabular}{|
>{\columncolor[HTML]{E6E6E6}}c |c|c|c|
>{\columncolor[HTML]{E6E6E6}}c |c|c|c|
>{\columncolor[HTML]{E6E6E6}}c |}
\hline
\rowcolor[HTML]{E6E6E6}
25 & 75 & 63 & 34 & 9 & 45 & 49 & 57 & 77 \\ \hline
3  & 58 & 6 & 51 & 4 & 27 & 39 & 18 & 76 \\ \hline
62 & 33 & 46 & 59 & 36                        & 11 & 50 & 5  & 42 \\ \hline
10 & 0  & 15 & 60 & 55                        & 35 & 74 & 28 & 14 \\ \hline
\rowcolor[HTML]{E6E6E6}
30 & 68 & 78 & 21 & 31                        & 29 & 54 & 73 & 65 \\ \hline
40 & 48 & 80 & 43 & 44                        & 38 & 37 & 23 & 72 \\ \hline
19 & 7  & 22 & 32 & \cellcolor[HTML]{CCCCCC}2 & 1  & 67 & 70 & 71 \\ \hline
64 & 61 & 56 & 8  & 79                        & 16 & 52 & 17 & 69 \\ \hline
\rowcolor[HTML]{E6E6E6}
13 & 47 & 12 & 20 & 66                        & 41 & 26 & 24 & 53 \\ \hline
\end{tabular}
\quad
\begin{tabular}{|
>{\columncolor[HTML]{E6E6E6}}c |c|c|
>{\columncolor[HTML]{E6E6E6}}c |
>{\columncolor[HTML]{E6E6E6}}c |c|c|
>{\columncolor[HTML]{E6E6E6}}c |}
\hline
\rowcolor[HTML]{E6E6E6}
16 & 8  & 25 & 24 & 22 & 38 & 37 & 53 \\ \hline
54 & 3  & 27 & 17 & 57 & 10 & 35 & 62 \\ \hline
4  & 28 & 5  & 52 & 34 & 56 & 42 & 18 \\ \hline
\rowcolor[HTML]{E6E6E6}
20 & 12 & 6  & 11 & 21 & 49 & 26 & 61 \\ \hline
\rowcolor[HTML]{E6E6E6}
46 & 33 & 2  & 60 & 40 & 41 & 58 & 14 \\ \hline
48 & 47 & 51 & 9  & 43 & 13 & 23 & 19 \\ \hline
45 & 7  & 30 & 63 & 36 & 50 & 0  & 29 \\ \hline
\rowcolor[HTML]{E6E6E6}
39 & 44 & 15 & 59 & 55 & 31 & \cellcolor[HTML]{CCCCCC}1  & 32 \\ \hline
\end{tabular}
}
\end{center}
We then check to see if the minimal height is a local minimum.\\
If it is, we return it, otherwise we recurse into the quadrant of its smallest neighbour, including our original boundary, since there must be a minimum in that direction as we would be following the slope of the curve. We do this as follows.
\begin{center}
\texttt{%
\begin{tabular}{|
>{\columncolor[HTML]{E6E6E6}}c |c|c|c|
>{\columncolor[HTML]{CCCCCC}}c |c|c|c|
>{\columncolor[HTML]{CCCCCC}}c |}
\hline
25 & \cellcolor[HTML]{E6E6E6}75 & \cellcolor[HTML]{E6E6E6}63 & \cellcolor[HTML]{E6E6E6}34 & \cellcolor[HTML]{E6E6E6}9  & \cellcolor[HTML]{E6E6E6}45 & \cellcolor[HTML]{E6E6E6}49 & \cellcolor[HTML]{E6E6E6}57 & \cellcolor[HTML]{E6E6E6}77 \\ \hline
3  & 58                         & 6                          & 51                         & \cellcolor[HTML]{E6E6E6}4  & 27                         & 39                         & 18                         & \cellcolor[HTML]{E6E6E6}76 \\ \hline
62 & 33                         & 46                         & 59                         & \cellcolor[HTML]{E6E6E6}36 & 11                         & 50                         & 5                          & \cellcolor[HTML]{E6E6E6}42 \\ \hline
10 & 0                          & 15                         & 60                         & \cellcolor[HTML]{E6E6E6}55 & 35                         & 74                         & 28                         & \cellcolor[HTML]{E6E6E6}14 \\ \hline
30 & \cellcolor[HTML]{E6E6E6}68 & \cellcolor[HTML]{E6E6E6}78 & \cellcolor[HTML]{E6E6E6}21 & 31                         & \cellcolor[HTML]{CCCCCC}29 & \cellcolor[HTML]{CCCCCC}54 & \cellcolor[HTML]{CCCCCC}73 & 65                         \\ \hline
40 & 48                         & 80                         & 43                         & 44                         & 38                         & \cellcolor[HTML]{CCCCCC}37 & 23                         & 72                         \\ \hline
19 & 7                          & 22                         & 32                         & 2                          & \cellcolor[HTML]{B3B3B3}1  & \cellcolor[HTML]{CCCCCC}67 & \cellcolor[HTML]{CCCCCC}70 & 71                         \\ \hline
64 & 61                         & 56                         & 8                          & 79                         & 16                         & \cellcolor[HTML]{CCCCCC}52 & 17                         & 69                         \\ \hline
13 & \cellcolor[HTML]{E6E6E6}47 & \cellcolor[HTML]{E6E6E6}12 & \cellcolor[HTML]{E6E6E6}20 & 66                         & \cellcolor[HTML]{CCCCCC}41 & \cellcolor[HTML]{CCCCCC}26 & \cellcolor[HTML]{CCCCCC}24 & 53                         \\ \hline
\end{tabular}
\quad
\begin{tabular}{|
>{\columncolor[HTML]{E6E6E6}}c |c|c|
>{\columncolor[HTML]{E6E6E6}}c |
>{\columncolor[HTML]{E6E6E6}}c |c|c|
>{\columncolor[HTML]{E6E6E6}}c |}
\hline
16 & \cellcolor[HTML]{E6E6E6}8  & \cellcolor[HTML]{E6E6E6}25 & 24 & 22                         & \cellcolor[HTML]{E6E6E6}38 & \cellcolor[HTML]{E6E6E6}37 & 53                         \\ \hline
54 & 3                          & 27                         & 17 & 57                         & 10                         & 35                         & 62                         \\ \hline
4  & 28                         & 5                          & 52 & 34                         & 56                         & 42                         & 18                         \\ \hline
20 & \cellcolor[HTML]{E6E6E6}12 & \cellcolor[HTML]{E6E6E6}6  & 11 & 21                         & \cellcolor[HTML]{E6E6E6}49 & \cellcolor[HTML]{E6E6E6}26 & 61                         \\ \hline
46 & \cellcolor[HTML]{E6E6E6}33 & \cellcolor[HTML]{E6E6E6}2  & 60 & \cellcolor[HTML]{CCCCCC}40 & \cellcolor[HTML]{CCCCCC}41 & \cellcolor[HTML]{CCCCCC}58 & \cellcolor[HTML]{CCCCCC}14 \\ \hline
48 & 47                         & 51                         & 9  & \cellcolor[HTML]{CCCCCC}43 & \cellcolor[HTML]{CCCCCC}13 & \cellcolor[HTML]{CCCCCC}23 & \cellcolor[HTML]{CCCCCC}19 \\ \hline
45 & 7                          & 30                         & 63 & \cellcolor[HTML]{CCCCCC}36 & \cellcolor[HTML]{CCCCCC}50 & \cellcolor[HTML]{B3B3B3}0  & \cellcolor[HTML]{CCCCCC}29 \\ \hline
39 & \cellcolor[HTML]{E6E6E6}44 & \cellcolor[HTML]{E6E6E6}15 & 59 & \cellcolor[HTML]{CCCCCC}55 & \cellcolor[HTML]{CCCCCC}31 & \cellcolor[HTML]{CCCCCC}1  & \cellcolor[HTML]{CCCCCC}32 \\ \hline
\end{tabular}
}
\end{center}
Using the master theorem, we can reduce the $n \times n$ matrix to a $\sim\frac{n}{2} \times \frac{n}{2}$ matrix with $O(n)$ queries.
\begin{align*}
    T(n) &= T(n/2) + cn\\
    T(n) &= T(n/4) + c\frac{n}{2} + cn\\
    T(n) &= T(n/8) + c\frac{n}{4} + c\frac{n}{2} + cn\\
    &\quad\quad\quad\quad\vdots\\
    T(n) &= T(1) + cn(1 + \frac{1}{2} + \frac{1}{4} + \ldots)\\
    T(n) &= T(1) + 2cn
\end{align*}
Thus, our algorithm is of $\Theta(n)$ time.
\end{enumerate}

\section*{Question 8}
\begin{align*}
    \text{let } &A(k) = k^{th} \text{smallest element of } A\\
    \text{let } &B(k) = k^{th} \text{smallest element of } B\\
    \text{let } &i \text{ be the current iteration (starting from 1)}\\
    \text{let } &s_f(i) = \lfloor\frac{n}{2^i}\rfloor, s_c(i) = \lceil\frac{n}{2^i}\rceil\\
    \text{let } &a = s_f(1), b = s_c(1)\\
    \text{let } &m_a = A(a + 1), m_b = B(b + 1)\\
    \text{let } &N \text{ be the set of } n \text{ elements such that } \forall x \in N \cap A, x < m_a \text{ and } \forall x \in N \cap B, x < m_b
\end{align*}
When all elements in $N$ are smaller than $m_a$ and $m_b$, the median will be the smallest of both $m_a$ and $m_b$.\\

\noindent
So we iterate.

\noindent
if $B(b) > m_a$ then there exist elements in $N$ that are not smaller than $m_a$:\\
    \indent Therefore, we change $m_a$ and $m_b$ and increment $i$.\\
    \indent We increase $a$ by $s_c(i)$ to consider elements that may be smaller than the median.\\
    \indent We decrease $b$ by $s_f(i)$ to ignore elements that may be larger than the median.\\
else if $A(a) > m_b$ then there exist elements in $N$ that are not smaller than $m_b$:\\
    \indent Therefore, we change $m_a$ and $m_b$ and increment $i$.\\
    \indent We decrease $a$ by $s_f(i)$ to ignore elements that may be larger than the median.\\
    \indent We increase $b$ by $s_c(i)$ to consider elements that may be smaller than the median.\\
else:\\
    \indent We stop iterating, as all elements of $N$ are smaller than both $m_a$ and $m_b$, and we take the median to be smaller of the two\\

\noindent This algorithm will take at most $O(\log n)$ queries, as every iteration we jump by a factor of 2, similar to that of a binary search.\\

\noindent
Note: if $a + 1 > n$ assume $m_b$ to be the median.

\end{document}
