\documentclass{article}
\usepackage{geometry}
\usepackage[utf8]{inputenc}
\usepackage{mathtools}
\usepackage{enumitem}
\usepackage{fancyhdr}
\usepackage{chngcntr}
\lhead{Evan Kohilas - z5114986}
\rhead{COMP3821 - Assignment 2}
\pagestyle{fancy}
\title{A17S1N1}
\counterwithin*{equation}{section}
\begin{document}
\begin{center}
    \begin{LARGE}
        COMP3121\\
        Assignment 2\\
        A17S1N2\\
        \hrulefill\\
        Evangelos Kohilas\\
        z5114986\\
        \hrulefill
    \end{LARGE}

    \begin{large}
        By submitting this document you are confirming that all the answers are your work and are not take from any other sources unless clearly mentioned.
    \end{large}

\end{center}

\section*{Question 1}
\begin{enumerate}[label=\alph*)]
    \item
        \begin{gather}
            \text{let } A^n
            = \begin{pmatrix} F(n+1) & F(n) \\ F(n) & F(n-1)  \end{pmatrix}
        \end{gather}
        When $n = 1$
        \begin{gather}
            A^1 = \begin{pmatrix} 1 & 1 \\ 1 & 0 \end{pmatrix}
        \end{gather}
        Assume $n = k$
        \begin{gather}
            A^k = \begin{pmatrix} F(k+1) & F(k) \\ F(k) & F(k-1)  \end{pmatrix}
        \end{gather}
        let $n = k + 1$
        \begin{align}
            A^{k+1} &= AA^k \\
            &= \begin{pmatrix} 1 & 1 \\ 1 & 0 \end{pmatrix}\begin{pmatrix} F(k+1) & F(k) \\ F(k) & F(k-1)\end{pmatrix} \\
                &= \begin{pmatrix} F(k+1) + F(k) & F(k) + F(k-1) \\ F(k+1) & F(k)  \end{pmatrix} \\
                    &= \begin{pmatrix} F(k+2) & F(k+1) \\ F(k+1) & F(k)  \end{pmatrix}
        \end{align}
        Therefore the forumla is true by induction for all $n > 0$.
    \pagebreak
    \item
        $F(n)$ can be found in $log_{2}(n)$ matrix multiplications using a recursive algorithm
        \begin{verbatim}
matrix = ((1, 1), (1,0))
func(n):
    if n == 1:
        return matrix
    if n is even:
        return func(n/2)^2
    if n is odd:
        return func(n-1) * matrix
        \end{verbatim}
\end{enumerate}

\section*{Question 2}
\begin{enumerate}[label=\alph*)]
    \item
        First we calculate the Karastuba trick.
        \setcounter{equation}{0}
        \begin{gather}
            (a + b)(c + d) = ac + ad + bd + bc \\
            ad + bd = (a + b)(c + d) - ac - bc
        \end{gather}
        Then we substitute $(2)$ into $(4)$
        \begin{align}
            (a + ib)(c + id) &= ac + adi + bdi + bc \\
            &= ac + i(ad + bd) + bc \\
            &= ac + i((a + b)(c + d) - ac - bc) + bc
        \end{align}
    \item
        \setcounter{equation}{0}
        First we calculate the Karastuba trick.
        \begin{gather}
            (a + b)^2 = a^2 + b^2 + 2ab \\
            2ab = (a + b)^2 - a^2 - b^2
        \end{gather}
        Then we substitute $(2)$ into $(3)$
        \begin{align}
            (a + ib)^2 &= a^2 + 2abi - b^2 \\
            &= a^2 + i((a + b)^2 - a^2 - b^2) - b^2
        \end{align}
    \item
        \setcounter{equation}{0}
        By re-arranging by the laws of exponents:
        \begin{gather*}
            (a+ib)^2(c+id)^2 = ((a+ib)(c+id))^2
        \end{gather*}
        thus from above, we then calculate the middle multiplication using
        3 real number multiplications, and then we find the square as above
        using 2 more real number multiplications.
\end{enumerate}

\pagebreak
\section*{Question 3}
Expand $P(x)$ and $Q(x)$ as follows.
\begin{align*}
    & P(x) = a_0 + x^{17}(a_{17} + a_{19}x^{2} + a_{21}x^4 + a_{23}x^6)\\
    & Q(x) = b_0 + x^{17}(b_{17} + b_{19}x^{2} + b_{21}x^4 + b_{23}x^6)
\end{align*}
let $y = x^2$ so that
\begin{align*}
    & R_a(y) = a_{17} + a_{19}y + a_{21}y^2 + a_{23}y^3\\
    & R_b(y) = b_{17} + b_{19}y + b_{21}y^2 + b_{23}y^3
\end{align*}
then
\begin{align*}
    & P(x)Q(x) = a_0b_0 + x^{17}(a_0R_b(x^2) + b_0R_a(x^2)) + x^{34}R_a(x^2)R_b(x^2)
\end{align*}
through brute force, we then calculate
\begin{align*}
    & a_0b_0 \text{ to require 1 multiplication}\\
    & a_0R_b(x^2) \text{ and } b_0R_a(x^2) \text{ to require 4 multiplcations each}
\end{align*}
then to multiply $R_a(x^2)R_b(x^2)$ (of degree 3), we require $2(3)+1 = 7$ multiplcations using the generalised Karatsuba method.\\
and so we get $1 + 4 + 4 + 7 = 16$ multiplications of large numbers.

\section*{Question 4}
As $P(x)$ has all 15 roots of unity, and $x^{15} - 1$ and $P(x)$ are both of the same degree and are both monic then it follows that
\begin{gather*}
P(x) = x^{15} -1
\end{gather*}

\section*{Question 5}
For any input $(a_0, a_1, a_2, ..., a_{2^n-1})$, we can describe $a_i$'s new position by converting $i$ to $n$ binary places and finding the reversed sequence.
e.g. $6 \rightarrow 110 \rightarrow 011 \rightarrow 3$

\section*{Question 6}
let
\begin{align}
    f_m &= \sum_{i+j=m} (j+1)q_jq_i \\
    p_j &= (j+1)q_j
\end{align}
then subsitute $(2)$ into $(1)$
\begin{align}
    & \sum_{i+j=m} p_jq_i = \vec{p} \ast \vec{q}
\end{align}
as $(3)$ is a linear convolution, $f_m$ can be computed in $O(nlog{n})$


\section*{Question 7}
\begin{enumerate}[label=\alph*)]
    \item Assuming the furtest distance from 1 corner to it's opposite corner, an inefficient algorithm would require (2+3)*2+(n-2)*4 such queries
    \item text b goes\\
        here
\end{enumerate}

\section*{Question 8}
text goes here

\end{document}
